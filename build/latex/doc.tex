%% Generated by Sphinx.
\def\sphinxdocclass{report}
\documentclass[letterpaper,10pt,russian]{sphinxmanual}
\ifdefined\pdfpxdimen
   \let\sphinxpxdimen\pdfpxdimen\else\newdimen\sphinxpxdimen
\fi \sphinxpxdimen=.75bp\relax
\ifdefined\pdfimageresolution
    \pdfimageresolution= \numexpr \dimexpr1in\relax/\sphinxpxdimen\relax
\fi
%% let collapsible pdf bookmarks panel have high depth per default
\PassOptionsToPackage{bookmarksdepth=5}{hyperref}

\PassOptionsToPackage{booktabs}{sphinx}
\PassOptionsToPackage{colorrows}{sphinx}

\PassOptionsToPackage{warn}{textcomp}
\usepackage[utf8]{inputenc}
\ifdefined\DeclareUnicodeCharacter
% support both utf8 and utf8x syntaxes
  \ifdefined\DeclareUnicodeCharacterAsOptional
    \def\sphinxDUC#1{\DeclareUnicodeCharacter{"#1}}
  \else
    \let\sphinxDUC\DeclareUnicodeCharacter
  \fi
  \sphinxDUC{00A0}{\nobreakspace}
  \sphinxDUC{2500}{\sphinxunichar{2500}}
  \sphinxDUC{2502}{\sphinxunichar{2502}}
  \sphinxDUC{2514}{\sphinxunichar{2514}}
  \sphinxDUC{251C}{\sphinxunichar{251C}}
  \sphinxDUC{2572}{\textbackslash}
\fi
\usepackage{cmap}
\usepackage[T1]{fontenc}
\usepackage{amsmath,amssymb,amstext}
\usepackage{babel}





\usepackage[Sonny]{fncychap}
\ChNameVar{\Large\normalfont\sffamily}
\ChTitleVar{\Large\normalfont\sffamily}
\usepackage{sphinx}

\fvset{fontsize=auto}
\usepackage{geometry}


% Include hyperref last.
\usepackage{hyperref}
% Fix anchor placement for figures with captions.
\usepackage{hypcap}% it must be loaded after hyperref.
% Set up styles of URL: it should be placed after hyperref.
\urlstyle{same}

\addto\captionsrussian{\renewcommand{\contentsname}{Contents:}}

\usepackage{sphinxmessages}
\setcounter{tocdepth}{1}



\title{Docs}
\date{июл. 24, 2023}
\release{}
\author{Dzerassa}
\newcommand{\sphinxlogo}{\vbox{}}
\renewcommand{\releasename}{}
\makeindex
\begin{document}

\ifdefined\shorthandoff
  \ifnum\catcode`\=\string=\active\shorthandoff{=}\fi
  \ifnum\catcode`\"=\active\shorthandoff{"}\fi
\fi

\pagestyle{empty}
\sphinxmaketitle
\pagestyle{plain}
\sphinxtableofcontents
\pagestyle{normal}
\phantomsection\label{\detokenize{index::doc}}


\sphinxstepscope


\chapter{Общие сведения}
\label{\detokenize{common:id1}}\label{\detokenize{common::doc}}

\section{Информационная система «Учет получателей пенсий» Социального фонда РЮО}
\label{\detokenize{common:id2}}
\sphinxAtStartPar
С помощью данной ИС Социальный фонд будет осуществлять регистрацию граждан и создание индивидуальных лицевых счетов с постоянным страховым номером при первичной их регистрации
в системе обязательного пенсионного страхования, организовывать и вести сведения о каждом пенсионере.

\sphinxAtStartPar
ИС позволит контролировать выплаты пенсий через почтовое отделение Республики, автоматически создавать списки выплаченых и не выплаченых пенсии, а также формировать акт сверки,
что значительно автоматизирует работу всех территориальных отделов фонда.

\sphinxAtStartPar
Главное окно программы содержит все объекты, с которыми может работать пользователь: территориальные отделения, реестры, пенсионные дела, список пользователей и т.д.
Справа находится боковое меню для перехода на соответстующую страницу.


\subsection{Наименование предприятий (объединений) разработчика и заказчика (пользователя) системы}
\label{\detokenize{common:id3}}
\sphinxAtStartPar
Заказчик \sphinxhyphen{} Социальный фонд Республики Южная Осетия
Разработчик \sphinxhyphen{} «Народные технологии»

\sphinxstepscope


\chapter{Назначение и цели создания системы}
\label{\detokenize{goals:id1}}\label{\detokenize{goals::doc}}

\section{Назначение системы}
\label{\detokenize{goals:id2}}\begin{description}
\sphinxlineitem{Основным назначением системы является:}\begin{itemize}
\item {} 
\sphinxAtStartPar
\sphinxstyleemphasis{регистрация пенсионных дел}

\item {} 
\sphinxAtStartPar
\sphinxstyleemphasis{ведение базы данных пенсионеров}

\item {} 
\sphinxAtStartPar
\sphinxstyleemphasis{формирование реестров пенсионеров}

\item {} 
\sphinxAtStartPar
\sphinxstyleemphasis{контроль выплат пенсий через почтовое отделение Республики}

\item {} 
\sphinxAtStartPar
\sphinxstyleemphasis{автоматическое формирование необходимых документов}

\end{itemize}

\end{description}


\section{Цели}
\label{\detokenize{goals:id3}}\begin{description}
\sphinxlineitem{Главной целью создания системы является:}\begin{itemize}
\item {} 
\sphinxAtStartPar
\sphinxstyleemphasis{создание общей базы данных пенсионеров}

\end{itemize}

\end{description}

\sphinxstepscope


\chapter{Требования к системе в целом}
\label{\detokenize{term:id1}}\label{\detokenize{term::doc}}
\sphinxAtStartPar
Архитектура системы должна подразумевать доступ
Система должна предоставить следующие возможности заказчику:
\begin{itemize}
\item {} 
\sphinxAtStartPar
\sphinxstyleemphasis{учет сотрудников заказчика, предоставление им возможности пользования системой},

\item {} 
\sphinxAtStartPar
\sphinxstyleemphasis{учет территориальных отделений},

\item {} 
\sphinxAtStartPar
\sphinxstyleemphasis{учет реестров на выплату пенсий},

\end{itemize}


\section{Учет сотрудников заказчика, предоставление им возможности пользования системой}
\label{\detokenize{term:id2}}\begin{description}
\sphinxlineitem{Учет сотрудников должен позволить хранить следующие сведениями:}\begin{itemize}
\item {} 
\sphinxAtStartPar
\sphinxstyleemphasis{Идентификатор сотрудника в системе},

\item {} 
\sphinxAtStartPar
\sphinxstyleemphasis{Фамилия \sphinxhyphen{} строка},

\item {} 
\sphinxAtStartPar
\sphinxstyleemphasis{Имя \sphinxhyphen{} строка},

\item {} 
\sphinxAtStartPar
\sphinxstyleemphasis{Отчество \sphinxhyphen{} строка},

\item {} 
\sphinxAtStartPar
\sphinxstyleemphasis{Дата рождения \sphinxhyphen{} дата}.

\end{itemize}

\sphinxlineitem{Учет сотрудников должен позволить:}\begin{itemize}
\item {} 
\sphinxAtStartPar
\sphinxstyleemphasis{добавлять сотрудника},

\item {} 
\sphinxAtStartPar
\sphinxstyleemphasis{вносить изменения в сведения о сотрудниках},

\item {} 
\sphinxAtStartPar
\sphinxstyleemphasis{просматривать список сотрудников},

\item {} 
\sphinxAtStartPar
\sphinxstyleemphasis{удаление сотрудников}.

\end{itemize}

\end{description}


\section{Учет территориальных отделений}
\label{\detokenize{term:id3}}\begin{description}
\sphinxlineitem{Учет территориальных отделений должен хранить следующие сведения:}\begin{itemize}
\item {} 
\sphinxAtStartPar
\sphinxstyleemphasis{идентификатор территориального отделения в системе},

\item {} 
\sphinxAtStartPar
\sphinxstyleemphasis{наименование территориального отделения},

\item {} 
\sphinxAtStartPar
\sphinxstyleemphasis{заметка}

\end{itemize}

\sphinxlineitem{Учет территориальных отделений должен позволить:}\begin{itemize}
\item {} 
\sphinxAtStartPar
\sphinxstyleemphasis{добавлять территориальное отделение},

\item {} 
\sphinxAtStartPar
\sphinxstyleemphasis{вносить изменения},

\item {} 
\sphinxAtStartPar
\sphinxstyleemphasis{просматривать список территориальных отделений},

\item {} 
\sphinxAtStartPar
\sphinxstyleemphasis{удалять территориальное отделение}.

\end{itemize}

\end{description}


\section{Учет реестров на выплату пенсий}
\label{\detokenize{term:id4}}\begin{description}
\sphinxlineitem{Вкладка реестры должна содержать следующие вкладки:}\begin{itemize}
\item {} 
\sphinxAtStartPar
\sphinxstyleemphasis{реестр},

\item {} 
\sphinxAtStartPar
\sphinxstyleemphasis{список пенсионеров},

\item {} 
\sphinxAtStartPar
\sphinxstyleemphasis{список описей}

\end{itemize}

\sphinxlineitem{Вкладка \sphinxstyleemphasis{реестр} должна содержать следующие сведения:}\begin{itemize}
\item {} 
\sphinxAtStartPar
\sphinxstyleemphasis{идентификатор реестра в системе},

\item {} 
\sphinxAtStartPar
\sphinxstyleemphasis{наименованеи реестра},

\item {} 
\sphinxAtStartPar
\sphinxstyleemphasis{дата формирования реестра},

\item {} 
\sphinxAtStartPar
\sphinxstyleemphasis{способ перечисления пенсии},

\item {} 
\sphinxAtStartPar
\sphinxstyleemphasis{статус реестра},

\item {} 
\sphinxAtStartPar
\sphinxstyleemphasis{количество выплаченных пенсий},

\item {} 
\sphinxAtStartPar
\sphinxstyleemphasis{общая сумма выплат},

\item {} 
\sphinxAtStartPar
\sphinxstyleemphasis{сумма выплаченных пенсий},

\item {} 
\sphinxAtStartPar
\sphinxstyleemphasis{сумма не выплаченных пенсий}

\end{itemize}

\sphinxlineitem{Вкладка \sphinxstyleemphasis{список пенсионеров} должна содержать сведения:}\begin{itemize}
\item {} 
\sphinxAtStartPar
\sphinxstyleemphasis{идентификатор пенсионера в системе},

\item {} 
\sphinxAtStartPar
\sphinxstyleemphasis{ФИО пенсионера},

\item {} 
\sphinxAtStartPar
\sphinxstyleemphasis{лицевой счет},

\item {} 
\sphinxAtStartPar
\sphinxstyleemphasis{паспортные данные пенсионера},

\item {} 
\sphinxAtStartPar
\sphinxstyleemphasis{адрес},

\item {} 
\sphinxAtStartPar
\sphinxstyleemphasis{размер пенсии},

\item {} 
\sphinxAtStartPar
\sphinxstyleemphasis{статус выплаты}

\end{itemize}

\sphinxAtStartPar
Также на вкладке должна быть возможность фильтрации пенсионеров по виду пенсии и по статусу выплаты.

\sphinxlineitem{Вкладка \sphinxstyleemphasis{список описей} должна содержать сведения:}\begin{itemize}
\item {} 
\sphinxAtStartPar
\sphinxstyleemphasis{идентификатор описи в системе},

\item {} 
\sphinxAtStartPar
\sphinxstyleemphasis{дата формирования описи},

\item {} 
\sphinxAtStartPar
\sphinxstyleemphasis{сумма выплаченных пенсий},

\item {} 
\sphinxAtStartPar
\sphinxstyleemphasis{количество выплаченных пенсий}

\end{itemize}

\end{description}



\renewcommand{\indexname}{Алфавитный указатель}
\printindex
\end{document}