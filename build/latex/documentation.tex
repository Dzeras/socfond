%% Generated by Sphinx.
\def\sphinxdocclass{report}
\documentclass[letterpaper,10pt,english]{sphinxmanual}
\ifdefined\pdfpxdimen
   \let\sphinxpxdimen\pdfpxdimen\else\newdimen\sphinxpxdimen
\fi \sphinxpxdimen=.75bp\relax
\ifdefined\pdfimageresolution
    \pdfimageresolution= \numexpr \dimexpr1in\relax/\sphinxpxdimen\relax
\fi
%% let collapsible pdf bookmarks panel have high depth per default
\PassOptionsToPackage{bookmarksdepth=5}{hyperref}

\PassOptionsToPackage{booktabs}{sphinx}
\PassOptionsToPackage{colorrows}{sphinx}

\PassOptionsToPackage{warn}{textcomp}
\usepackage[utf8]{inputenc}
\ifdefined\DeclareUnicodeCharacter
% support both utf8 and utf8x syntaxes
  \ifdefined\DeclareUnicodeCharacterAsOptional
    \def\sphinxDUC#1{\DeclareUnicodeCharacter{"#1}}
  \else
    \let\sphinxDUC\DeclareUnicodeCharacter
  \fi
  \sphinxDUC{00A0}{\nobreakspace}
  \sphinxDUC{2500}{\sphinxunichar{2500}}
  \sphinxDUC{2502}{\sphinxunichar{2502}}
  \sphinxDUC{2514}{\sphinxunichar{2514}}
  \sphinxDUC{251C}{\sphinxunichar{251C}}
  \sphinxDUC{2572}{\textbackslash}
\fi
\usepackage{cmap}
\usepackage[T1]{fontenc}
\usepackage{amsmath,amssymb,amstext}
\usepackage{babel}



\usepackage{tgtermes}
\usepackage{tgheros}
\renewcommand{\ttdefault}{txtt}



\usepackage[Bjarne]{fncychap}
\usepackage{sphinx}

\fvset{fontsize=auto}
\usepackage{geometry}


% Include hyperref last.
\usepackage{hyperref}
% Fix anchor placement for figures with captions.
\usepackage{hypcap}% it must be loaded after hyperref.
% Set up styles of URL: it should be placed after hyperref.
\urlstyle{same}

\addto\captionsenglish{\renewcommand{\contentsname}{Contents:}}

\usepackage{sphinxmessages}
\setcounter{tocdepth}{1}



\title{Documentation}
\date{Apr 20, 2023}
\release{}
\author{dzerassa}
\newcommand{\sphinxlogo}{\vbox{}}
\renewcommand{\releasename}{}
\makeindex
\begin{document}

\ifdefined\shorthandoff
  \ifnum\catcode`\=\string=\active\shorthandoff{=}\fi
  \ifnum\catcode`\"=\active\shorthandoff{"}\fi
\fi

\pagestyle{empty}
\sphinxmaketitle
\pagestyle{plain}
\sphinxtableofcontents
\pagestyle{normal}
\phantomsection\label{\detokenize{index::doc}}


\sphinxstepscope


\chapter{Требования к системе в целом}
\label{\detokenize{term:id1}}\label{\detokenize{term::doc}}
\sphinxAtStartPar
Архитектура системы должна подразумевать доступ
Система должна предоставить следующие возможности заказчику:
* учет сотрудников заказчика, предоставление им возможности пользования системой,
* учет территориальных отделений,
* учет реестров на выплату пенсий.


\section{Учет сотрудников заказчика, предоставление им возможности пользования системой}
\label{\detokenize{term:id2}}
\sphinxAtStartPar
Учет сотрудников должен позволить хранить следующие сведениями:
* Идентификатор сотрудника в системе,
* Фамилия \sphinxhyphen{} строка,
* Имя \sphinxhyphen{} строка,
* Отчество \sphinxhyphen{} строка,
* Дата рождения \sphinxhyphen{} дата.

\sphinxAtStartPar
Учет сотрудников должен позволить:
* добавлять сотрудника,
* вносить изменения в сведения о сотрудниках,
* просматривать список сотрудников,
* удаление сотрудников.


\section{Добавление сотрудника}
\label{\detokenize{term:id3}}
\sphinxAtStartPar
Должна быть реализована возможность добавлять сотрудника в реестр сотрудников.


\section{Учет территориальных отделений}
\label{\detokenize{term:id4}}
\sphinxAtStartPar
Учет территориальных отделений должен хранить следующие сведения:
* идентификатор территориального отделения в системе,
* наименование территориального отделения,
* просмотр сотрудников территориального отделения.

\sphinxAtStartPar
Учет территориальных отделений должен позволить:
* добавлять территориальное отделение,
* вносить изменения,
* просматривать список территориальных отделений,
* удалять территориальное отделение.


\section{Учет реестров на выплату пенсий}
\label{\detokenize{term:id5}}
\sphinxAtStartPar
Учет реестров на выплату пенсий должен содержать следующие сведения:
* идентификатор реестра в системе,
* наименованеи реестра,
*

\sphinxstepscope


\chapter{Назначение и цели создания системы}
\label{\detokenize{goals:id1}}\label{\detokenize{goals::doc}}

\section{Назначение системы}
\label{\detokenize{goals:id2}}
\sphinxAtStartPar
Система предназначена для учета сведений о пенсионерах Республики Южная Осетия.
Система позволит вести реестры пенсионеров.


\section{Цели}
\label{\detokenize{goals:id3}}
\sphinxAtStartPar
Система позволить ускорить

\sphinxstepscope


\chapter{Общие сведения}
\label{\detokenize{common:id1}}\label{\detokenize{common::doc}}

\section{Информационная система “Учет получателей пенсий” Социального фонда РЮО}
\label{\detokenize{common:id2}}

\subsection{Наименование предприятий (объединений) разработчика и заказчика (пользователя) системы}
\label{\detokenize{common:id3}}
\sphinxAtStartPar
Заказчик \sphinxhyphen{} Социальный фонд Республики Южная Осетия
Разработчик \sphinxhyphen{} “Народные технологии”



\renewcommand{\indexname}{Index}
\printindex
\end{document}